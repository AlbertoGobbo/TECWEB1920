
\paragraph{Visualizzazione Homepage del sito}\mbox{}\\
\label{par:VisHome}
L'utente può entrare nella Homepage in diversi modi:
\begin{itemize}
	\item Se è appena entrato nel sito è la prima pagina che viene visualizzata;
	\item Se si trova in un'altra pagina, può raggiungere la Homepage cliccando la sezione "Home" nella navbar;
	\item Se si trova in un'altra pagina, può raggiungere la Homepage cliccando sopra il logo presente nell'header;
\end{itemize}
All'interno di questa pagina l'utente può visualizzare una breve descrizione della pasticceria,
le ultime news, gli orari e può sapere la sua ubicazione.

\paragraph{Visualizzazione dei prodotti in vendita}\mbox{}\\
\label{par:Visprodotti}
L'utente entra nella pagina di visualizzazione dei prodotti cliccando la sezione paste o torte presente nella navbar.\\
L'utente inoltre potrà anche lasciare una recensione (una sola per ogni prodotto).

\paragraph{Visualizzazione pagina "La storia"}\mbox{}\\
\label{par:VisStoria}
Attraverso questa pagina l'utente potrà visualizzare la storia della Pasticceria Padovana.\\
Vengono fornite informazioni riguardanti l'anno di apertura, l'evoluzione del laboratorio,
nonchè dei nostri pasticceri che da sempre deliziano il palato di moltissimi clienti.

\paragraph{Visualizzazione pagina "Chi siamo"}\mbox{}\\
\label{par:VisAbout}
L'utente entra nella pagina di "chi siamo" cliccando l'apposita sezione presente nella navbar.\\
All'interno di questa pagina statica sono contenute le informazioni riguardanti i nostri pasticceri,
gli orari, l'ubicazione e la possibilità di contatto.

\paragraph{Registrazione}\mbox{}\\
\label{par:Reg}
L'utente entra nella pagina di registrazione cliccando la sezione "Registrati" presente nell'header.\\
La pagina offre all'utente la possibilità di creare il proprio account personale. \\
Per effettuare la registrazione l'utente deve inserire una serie di dati,
 alcuni obbligatori (quelli contrassegnati con *) e altri opzionali. 

\paragraph{Login}\mbox{}\\
\label{par:Login}
L'utente entra nella pagina di login cliccando la sezione "Accedi" presente nell'header".\\
La pagina offre all'utente già in possesso di un account la possibilità di accedervi, inserendo le proprie credenziali (email e password).


