Il sito Internet si propone di fornire informazioni riguardanti la Pasticceria Padovana, ed è pensato per garantire una navigazione fluida con il minor numero di operazioni, per guardare e cercare i prodotti offerti dalla pasticceria.\\
Pertanto gli utilizzatori del sito appartengono ad una categoria di utenti omogenei: dai possibili clienti della pasticceria ai visitatori casuali del sito, fino ad arrivare ai dipendenti dell'attività.\\
Da evidenziare che questa categoria di utenti è esterna al sito è verrà denonimanata con la dicitura \emph{utente generico}.
L'unico utente che può attivare le funzionalità speciali del sito e che può essere considerato a tutti gli effetti un utente interno, viene chiamato \emph{amministratore} e lo diventa a tutti gli effetti dopo aver provveduto ad accedere alla zona riservata del sito tramite un form dedicato.\\
Quindi, l'utente finale è principalmente generico, e per questo è necessario utilizzare un linguaggioinformale, semplice e di facile intuizione, che possa essere compreso dalla maggior parte delle persone.\\ 
Lo stesso concetto vale per la struttura del sito e per il layout, che hanno l'obiettivo di fornire un modello mentale più familiare possibile all'utente, evitando di rompere le convenzioni esterne e, indipendentemente dalla strategia di browsing, rendendo veloce e intuitiva la navigazione all'interno del sito.\\