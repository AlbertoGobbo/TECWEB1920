L'utente viene definito \emph{generico} nel momento in cui può solo navigare all'interno del sito, senza alcun permesso di accedere all'area riservata.\\
L'utente generico dispone dei seguenti casi d'uso:
\begin{itemize}
	\item \hyperref[par:VisHome]{ Visualizzazione pagina "Home" (\ref{par:VisHome})};
	\item \hyperref[par:VisPaste]{ Visualizzazione pagina "Paste"(\ref{par:VisPaste})};
	\item \hyperref[par:CerPaste]{ Ricerca prodotti in "Paste" (\ref{par:CerPaste})};
	\item \hyperref[par:PrevNextPaste]{ Scorrimento pagine in "Paste" (\ref{par:PrevNextPaste})};
	\item \hyperref[par:VisTorte]{ Visualizzazione pagina "Torte" (\ref{par:VisTorte})};
	\item \hyperref[par:CerTorte]{ Ricerca prodotti in "Torte" (\ref{par:CerTorte})};
	\item \hyperref[par:PrevNextTorte]{ Scorrimento pagine in "Torte" (\ref{par:PrevNextTorte})};
	\item \hyperref[par:VisStoria]{ Visualizzazione pagina "Storia" (\ref{par:VisStoria})};
	\item \hyperref[par:VisContatti]{ Visualizzazione pagina "Contatti" (\ref{par:VisContatti})};
\end{itemize}

\paragraph{Visualizzazione pagina "Home"}\mbox{}\\
\label{par:VisHome}
L'utente generico può entrare nella pagina \emph{Home} in diversi modi:
\begin{itemize}
	\item se è appena entrato nel sito, è la prima pagina che viene visualizzata;
	\item se si trova in un'altra pagina, può raggiungere la homepage cliccando la sezione "Home" presente nel menu laterale;
	\item se si trova in un'altra pagina, può raggiungere la homepage cliccando sul logo presente nell'header;
\end{itemize}
All'interno di questa pagina l'utente può visualizzare una breve presentazione della pasticceria.\\

\paragraph{Visualizzazione pagina "Paste"}\mbox{}\\
\label{par:VisPaste}
L'utente generico può entrare nella pagina \emph{Paste} cliccando la sezione "Paste" presente nel menu laterale.
In questa pagina vengono visualizzate al massimo 7 paste alla volta. Ogni pasta conterrà un'immagine, il titolo e una descrizione.\\ 

\paragraph{Ricerca prodotti in "Paste"}\mbox{}\\
\label{par:CerPaste}
Una volta entrato nella pagina \emph{Paste}, si potrà visualizzare una barra di ricerca posta all'inizio del content.\\
L'utente può cercare il prodotto desiderato inserendo il titolo della pasta o una sottostringa del titolo stesso all'interno della barra di ricerca. Cliccando sul bottone 
\emph{cerca} visualizzerà istantaneamente le paste desiderate.

\paragraph{Scorrimento pagine in "Paste"}\mbox{}\\
\label{par:PrevNextPaste}
Una volta entrato nella pagina \emph{Paste}, l'utente può scrollare tutta la pagina arrivando fino alla fine del contenuto.\\
Sono due le possibilità che si possono verificare:
\begin{enumerate}
	\item se si trova nella prima pagina e ci sono meno di 7 paste nel database, non si visualizzerà alcuna modifica grafica alla pagina;
	\item altrimenti tra l'ultimo prodotto visualizzabile e il footer, nel content sarà presente un bottone \emph{Pagina successiva} che offre la possibilità di andare avanti 
	di una pagina.
\end{enumerate}	
Ammesso che si verifichi il secondo caso, alla pagina successiva possono verificarsi due casi:
\begin{itemize}
	\item se rimangono ancora più di 7 prodotti da visualizzare, allora si presenterà il bottone \emph{Pagina successiva} affiancato alla sua sinistra dal 
	bottone \emph{Pagina precedente}, che offre la possibilità di tornare indietro di una pagina;
	\item in caso contrario, si visualizzerà solo il bottone \emph{Pagina precedente}.
\end{itemize}	

\paragraph{Visualizzazione pagina "Torte"}\mbox{}\\
\label{par:VisTorte}
L'utente generico può entrare nella pagina \emph{Torte} cliccando la sezione "Torte" presente nel menu laterale.
In questa pagina vengono visualizzate al massimo 7 torte alla volta. Ogni pasta conterrà un'immagine, il titolo e una descrizione.\\

\paragraph{Ricerca prodotti in "Torte"}\mbox{}\\
\label{par:CerTorte}
Una volta entrato nella pagina \emph{Torte}, si potrà visualizzare una barra di ricerca posta all'inizio del content.\\
L'utente può cercare il prodotto desiderato inserendo il titolo della pasta o una sottostringa del titolo stesso all'interno della barra di ricerca. Cliccando sul bottone 
\emph{cerca} visualizzerà instantaneamente le torte desiderate.

\paragraph{Scorrimento pagine in "Torte"}\mbox{}\\
\label{par:PrevNextTorte}
Una volta entrato nella pagina \emph{Torte}, l'utente può scrollare tutta la pagina arrivando fino alla fine del contenuto.\\
Sono due le possibilità che si possono verificare:
\begin{enumerate}
	\item se si trova nella prima pagina e ci sono meno di 7 torte nel database, non si visualizzerà alcuna modifica grafica alla pagina;
	\item altrimenti tra l'ultimo prodotto visualizzabile e il footer, nel content sarà presente un bottone \emph{Pagina successiva} che offre la possibilità di andare avanti 
	di una pagina.
\end{enumerate}	
Ammesso che si verifichi il secondo caso, alla pagina successiva possono verificarsi due casi:
\begin{itemize}
	\item se rimangono ancora più di 7 prodotti da visualizzare, allora si presenterà il bottone \emph{Pagina successiva} affiancato alla sua sinistra dal 
	bottone \emph{Pagina precedente}, che offre la possibilità di tornare indietro di una pagina;
	\item in caso contrario, si visualizzerà solo il bottone \emph{Pagina precedente}.
\end{itemize}	

\paragraph{Visualizzazione pagina "Storia"}\mbox{}\\
\label{par:VisStoria}
Attraverso questa pagina l'utente potrà visualizzare la storia della Pasticceria Padovana.\\
Vengono fornite informazioni riguardanti la nascita dell'azienda e l'evoluzione del laboratorio, nonchè dei nostri pasticceri che da sempre deliziano il palato di moltissimi clienti.\\
Non mancano riferimenti ai prestigiosi premi vinti nel corso degli anni.

\paragraph{Visualizzazione pagina "Contatti"}\mbox{}\\
\label{par:VisContatti}
L'utente generico può entrare nella pagina \emph{Contatti} in diversi modi: se si trova in "Home","Paste","Torte","Storia"
\begin{itemize}
	\item cliccando la sezione "Contatti" presente nel menu laterale; 
	\item cliccando sul link \emph{Vieni a trovarci!} presente in un box contenente anche gli orari di apertura, posto sotto il box delle news che a sua volta è posto sotto il menu;
\end{itemize}
All'interno di questa pagina sono contenute le informazioni relative ai vari contatti aziendali (telefono e mail),
gli orari di apertura e l'ubicazione dell'attività su Google Maps.
