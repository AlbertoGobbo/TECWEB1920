Il gruppo ha deciso di elaborare le pagine server-side tramite il linguaggio \emph{PHP}. Questa operazione è necessaria, in quanto ogni 
pagina deve specializzarsi in base al tipo di utenza (distinzione tra utente generico ed amministratore).\\
Il linguaggio \emph{PHP} consente di leggere e scrivere nel database, oltre che eseguire operazioni sul filesystem, come inserimento e 
cancellazione delle foto dei prodotti.\\
Il sito Internet è principalmente strutturato in questo modo: 
\begin{itemize}
    \item un file \emph{template.html} contenente tutte le parti statiche condivise da tutte le pagine;
    \item alcuni file \emph{'pagina'.php}, i quali elaborano il contenuto del template e ritornano la pagina modificata;
    \item i file \emph{storia.html, home.html} e \emph{contatti.html} usati dalle corrispondenti pagine php per recuperare le parti statiche;
    \item classi PHP per l'elaborazione e stampa delle parti dinamiche del sito, ad esempio il menù e la pagina dei prodotti;
    \item classi PHP per l'elaborazione di dati per il database o il salvataggio (o eliminazione) delle immagini;
\end{itemize}
In questo modo, si mantiene la separazione tra struttura e comportamento, in quanto la struttura viene descritta in file XHTML, mentre il comportamento
è compito esclusivo di file PHP. \\
Il ruolo del \emph{PHP} non sarà modificare la struttura della pagina, che sarà compito esclusivo di \emph{XHTML}, ma sarà elaborare 
informazioni e modificare quello che è il contenuto della pagina richiesta.\\

Per soddisfare un'altra importante richiesta del progetto, ovvero \emph{"deve essere presente una forma di controllo dell’input inserito dall’utente, 
sia lato client che lato server"}, sono stati implementati controlli PHP (server-side) sugli input dei form presenti nel sito, che corrispondono ai 
controlli client-side attuati con JavaScript/JQuery.\\
Questo potrebbe risultare inutile, ma in realtà è un ulteriore controllo di sicurezza fondamentale. Basti pensare banalmente che un utente
avanzato potrebbe alterare il comportamento client-side per oltrepassare i controlli. Purtroppo per lui, e per fortuna del gestore e degli
utilizzatori del sito, l'utente che ha cercato di infiltrarsi non potrà mai modificare il comportamento \emph{PHP}, in quanto è codice server-side.\\
Per questo è fornita una classe \emph{Input\_security\_check} contenente tutti i metodi necessari ai vari controlli, che operano controllando 
i singoli caratteri in input.
\begin{itemize}
    \item Il metodo \emph{general\_controls(input)} esegue operazioni fondamentali per quanto riguarda la sicurezza. Dato un input, il metodo si 
    occupa di passarlo tramite una serie di tre funzioni: 
    \begin{itemize}
        \item \emph{trim()}: elimina gli spazi prima e dopo la stringa in input;
        \item \emph{htmlentities()}: converte tutti i possibili caratteri in entità \emph{HTML};
        \item \emph{strip\_tags()}: elimina tutti i possibili tag \emph{HTML} all'interno;
    \end{itemize}
    \item I metodi \emph{username\_check()} e \emph{password\_check()}, dopo aver richiamato il metodo di controllo generale, si occupano di 
    verificare se i possibili caratteri in input sono stati rispettati;
    \item Il metodo \emph{general\_input\_check()}, dopo aver richiamato il metodo di controllo generale, si occupa di aggiungere uno slash a 
    tutti i caratteri che potrebbero influire all'esecuzione di una query tramite la funzione \emph{addslashes()};
\end{itemize}

Ognuno di questi metodi si occuperà anche di verificare la lunghezza dell'input: ogni campo inserito ha una lunghezza minima e massima.\\