Il gruppo ha deciso di elaborare le pagine server-side tramite il linguaggio \emph{PHP}. Questa operazione è necessaria, in quanto ogni pagina deve specializzarsi in base 
al tipo di utenza.\\
Il linguaggio \emph{PHP} consente di leggere e scrivere nel database, oltre che eseguire operazioni sul filesystem quali inserimento e cancellazione delle foto dei prodotti.\\
Importante è sottolineare la separazione tra struttura e comportamento, il cui significato e dimostrazione sarà nelle prossime righe.\\
Il sito internet è principalmente strutturato in questo modo: 
\begin{itemize}
    \item un file \emph{template.html} contenente tutte le parti statiche condivise da tutte le pagine;
    \item dei file \emph{'pagina'.php} le quali elaborano il contenuto del template in infine ritornano la pagina modificata;
    \item dei file \emph{storia.html, home.html} e \emph{contatti.html} usati dalle corrispondenti pagine php per recuperare le parti statiche;
    \item classi php per l'elaborazione e stampa delle parti più dinamiche, come il menù o la pagina dei prodotti;
    \item classi php per l'elaborazione di dati per il database o il salvataggio (o eliminazione) delle immagini;
\end{itemize}
Così facendo il ruolo del \emph{PHP} non sarà quello di andare a modificare la struttura della pagina, ma quel compito verrà lasciato ad \emph{XHTML}.\\
Il \emph{PHP} si occupa di elaborare informazioni e modificare quello che è il contenuto della pagina, ma non la sua struttura.\\


Viene lasciato al lato server un ulteriore controllo di sicurezza sugli input dell'utente. Banalmente, un utente avanzato potrebbe alterare il comportamento
del \emph{JavaScript} per oltrepassare i controlli, ma non potrà mai modificare il comportamento \emph{PHP} poichè è codice server-side.\\
Per questo è fornita una classe \emph{Input\_security\_check} contenente tutti i metodi necessari ai vari controlli, che operano controllando i singoli caratteri in input.
\begin{itemize}
    \item Il metodo \emph{general\_controls(input)} esegue operazioni fondamentali per quanto riguarda la sicurezza. Dato un input il metodo si occupa di passarlo tramite una serie di tre funzioni: 
    \begin{itemize}
        \item \emph{trim()}: elimina gli spazi prima e dopo la stringa in input;
        \item \emph{htmlentities()}: converte tutti i possibili caratteri in entità \emph{HTML};
        \item \emph{strip\_tags()}: elimina tutti i possibili tag \emph{HTML} all'interno;
    \end{itemize}
    \item I metodi \emph{username\_check()} e \emph{password\_check}, dopo aver richiamato il metodo di controllo generale si occupano di verificare se i possibili caratteri in input sono stati rispettati;
    \item Il metodo \emph{general\_input\_check()} dopo aver richiamato il metodo di controllo generale si occupa di aggiungere uno slash a tutti i caratteri che potrebbero influire all'esecuzione di una query tramite la funzione \emph{addslashes()};
\end{itemize}

Test di query injection saranno descritti nella sezione test.