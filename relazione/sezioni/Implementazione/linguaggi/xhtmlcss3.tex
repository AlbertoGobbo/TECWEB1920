Il sito \emph{Pasticceria Padovana} è stato pensato per ogni tipo di utente, che sia all'avanguardia con le tecnologie 
e quindi sempre conforme agli aggiornamenti, oppure retrogrado tecnologicamente.\\
Per questo motivo, la scelta del linguaggio da utilizzare per il contenuto del sito è ricaduta su \emph{XHTML},
che ha una sintassi più rigida in fase di implementazione, in modo da garantire il funzionamento totale del sito nei browser esistenti 
e nelle loro versioni precedenti.\\
Per la realizzazione del sito non è stato necessario utilizzare funzionalità aggiuntive di \emph{HTML5}, linguaggio più recente ma che può causare
problemi di incompatibilità con alcuni browser.\\
Per i dettagli implementativi sono state seguite le linee guida del \emph{W3C} e le spiegazioni dei professori in aula.\\
Il linguaggio di stile utilizzato per la definizione della presentazione è \emph{CSS3}. 
Per mantenere la separazione tra contenuto e presentazione, sono state attuate le rispettive direttive: 
\begin{itemize}
    \item mantenendo i compiti dei due linguaggi separati, ovvero non utilizzando tag di stile in XHTML;
    \item separando completamente XHTML e CSS3, senza implementare il codice a cascata inline o embedded ad XHTML, ma solo su file esterni.
\end{itemize}
Ogni singola regola è stata valutata prima di utilizzarla, in base alla compatibilità dei browser.\\
Qualche regola CSS3 non è compatibile con IE8, ma è stata applicata in modo da ottenere una trasfomazione elegante.


