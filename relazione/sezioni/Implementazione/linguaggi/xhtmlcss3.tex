Il sito da noi realizzato è stato pensato per un'utenza assolutamente non all'avanguardia per quanto riguarda la tecnologia.\\
Proprio per questo è stato scelto di utilizzare \emph{XHTML}, in modo da garantire il funzionamento totale del sito in ogni ambiente, ma anche in modo da
avere una sintassi più rigida in fase di implementazione.\\ Inoltre, per la realizzazione non è stato necessario utilizzare funzionalità aggiuntive
di \emph{HTML5}, quindi il passaggio ad HTML5 mantenendo la sintassi \emph{XHTML} non è stato necessario.\\
Sono state seguite le linee guida del \emph{W3C}, oltre che le spiegazioni in aula.\\
Il linguaggio di stile utilizzato è il \emph{CSS3}. Molto importante da sottolineare è il mantenimento di una separazione tra struttura e presentazione, che 
si manifesta: 
\begin{itemize}
    \item mantenendo i compiti dei due linguaggi separati, ovvero non utilizzando tag di stile in XHTML;
    \item separando il più possibile XHTML e CSS3, quindi senza implementare il codice a cascata inline o embedded ad XHTML, ma solo su file esterni.
\end{itemize}
Ogni singola regola CSS2 o CSS3 è stata valutata prima di utilizzarla, in base alla compatibilità dei browser.
Qualche regola CSS3 non è compatibile con IE8 causando una trasfomazione elegante.


