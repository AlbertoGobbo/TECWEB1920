\emph{JavaScript} è un linguaggio che si occupa del comportamento del sito. Nel nostro caso specifico, il suo principale uso è rivolto dai controlli 
input lato client, dove ognuno di essi viene controllato prima di essere spedito, in modo da evitare elaborazioni inutili da parte del server.\\
Va evidenziato che i controlli client-side sono speculari ai controlli server-side.\\
L'utilizzo di JavaScript è importante anche per le animazioni del sito. Nel nostro progetto, le uniche animazioni da gestire sono implementate 
lato mobile, in quanto sono presenti alcuni elementi a comparsa.\\
La motivazione di questa scelta è molto semplice ed è giustificata dalla dimensione dello schermo, dove uno schermo mobile è più piccolo di uno schermo
in modalità desktop, quindi gli elementi da considerare devono occupare poco spazio, non essere invasivi ed essere essenziali ai fini dell'utente. \\
Quindi, è stato scelto di implementare:
\begin{itemize}
    \item un menù ad hamburger che mostra/nasconde le singole voci, rimpiazzando così il menu laterale lato desktop;
    \item un bottone \emph{Accesso amministratore} nel footer che, una volta premuto, fa comparire il form di login per l'amministratore, e per farlo
    scomparire può premere sul bottone \emph{Esci.} 
\end{itemize}    

La validazione dell'utente nel form di login avviene attraverso \emph{AJAX}, acronimo di \emph{Asynchronous JavaScript and XML}.\\ 
La motivazione di tale scelta è per il messaggio d'errore in caso di input scorretto: il form di login si trova nel footer, e nel caso di errore 
verrebbe inserito un messaggio d'errore dopo il ricaricamento della pagina, il che è scomodo perchè non si avrebbe accesso diretto ai campi del form
se non scrollando la pagina.\\
Grazie ad AJAX, l'errore viene visualizzato all'interno del form e la pagina non viene ricaricata.\\
Nel caso i vecchi browser non riescano a supportare questa funzionalità, visualizzeranno comunque un messaggio di errore ad inizio pagina, 
dopo il caricamento.\\
AJAX entra in funzione nel momento in cui avviene un click nel submit. L'evento, reso asincrono, si occuperà in primo luogo di controllare 
se le credenziali sono corrette: se non lo sono, viene disabilitato il comportamento standard del submit, per visualizzare 
un messaggio d'errore.\\