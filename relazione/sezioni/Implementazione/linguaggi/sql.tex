Per il salvataggio dei dati, il gruppo ha deciso di utilizzare SQL tramite \emph{MariaDB}.\\
Per eseguire le interrogazioni o aggiornare il database, è stata usata la libreria \emph{mysqli} di \emph{PHP}.\\
Il database collegato al sito è molto semplice, ed è descritto nel capitolo 3: \emph{Progettazione -> Database}.
\\
Da notare le lunghezze dei campi SQL più grandi rispetto a quelle richieste in input (questo nel caso delle news e dei prodotti).\\
La scelta è stata fatta poichè l'uso di un marcatore (ex.[en=testo]) per evidenziare una lingua diversa, o per evidenziare una parola o testo importante,
nel database viene tradotto in tag html e quest'ultimo ha una lunghezza maggiore.
