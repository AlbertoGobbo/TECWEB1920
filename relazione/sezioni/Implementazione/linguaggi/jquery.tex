Un primo uso di JQuery è occupato dai controlli input lato client. Ogni input utente viene controllato prima di essere spedito, in modo da non mandare lavoro inutile al server.\\
DA CONTINUARE DOPO CONTROLLI JS
Importante è far notare che, da lato mobile, viene implementato un uso maggiore di JQuery per poter inserire elementi a comparsa. Questo poichè lo spazio a schermo è molto inferiore
rispetto al desktop. Infatti è stato scelto di implementare un menù ad hamburger, e un bottone per la comparsa del form di login per l'admin, nel footer.\\

La validazione dell'utente, nel form di login, avviene attraverso Ajax. Questo poichè il form di login si trova nel footer, e viene scomodo inserire un messaggio di login
errato dopo il ricaricamento della pagina. I vecchi browser che non riescono a supportare questa funzionalità visualizzeranno comunque un messaggio di errore sopra la pagina, dopo
il caricamento.\\
Ajax entra in funzione nel momento in cui avviene un click nel submit. L'evento, reso asincrono, si occuperà in primis di controllare se le credenziali sono corrette o meno,
se non sono corrette, viene disabilitato il comportamento standard del submit, per visualizzare un messaggio di errore.\\