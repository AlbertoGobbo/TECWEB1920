Il progetto \emph{Pasticceria Padovana} si prefigge di rispettare le specifiche tecniche richieste, in particolare:
\begin{itemize}
	\item \textbf{Separazione tra contenuto, presentazione e comportamento:} primo requisito fondamentale. 
	Il contenuto è stato progettato in modo statico in XHTML 1.0, la presentazione in CSS3 e il comportamento è gestito in modo dinamico con PHP7 e JavaScript.\\
	E' stato scelto di non progettare il contenuto in HTML5 per il fatto che HTML5 non ha raggiunto la piena compatibilità con i browser attualmente in circolazione.\\
	\item \textbf{Accessibilità:} tutte le categorie di utenti devono poter utilizzare il sito senza particolari ostacoli.\\ 
	Il sito è stato costruito per organizzare i propri contenuti in modo da poter essere facilmente reperiti da qualsiasi utente, 
	in modo da non generare disorientamento e sovraccarico cognitivo.\\ 
	Le implementazioni più importanti sono:
	\begin{itemize}
		\item attributo \emph{alt} nelle immagini, che rende disponibile un testo alternativo in caso in cui l'immagine non venga caricata.\\ 
		Il testo alternativo deve essere coerente con l'immagine visualizzata: se è un'immagine decorativa e priva di significato si inserisce un testo vuoto, 
		altrimenti si scrive una breve descrizione.
		\item link \emph{Salta il contenuto}, nascosto di default ma visibile agli screen reader. E' la prima voce del menu, viene utilizzata nel caso in cui l'utente che
		utilizza lo screen reader voglia saltare direttamente al contenuto della pagina, velocizzando la navigazione.
		\item link {torna su}, sotto forma di bottone circolare sempre visibile, che si trova a lato del content: se invocato, ti porta all'inizio del content. 
		Utile per evitare lo scroll e velocizzare la navigazione.
		\item utilizzo della tecnologia \emph{WAI-ARIA}: abbinata ad XHTML nel doctype, è stata implementata per migliorare la lettura degli screen reader.\\
		I comandi più utilizzati sono: \emph{aria-hidden="true"}, che impedisce che un elemento venga letto dal lettore di schermo, e \emph{aria-label="..."}, 
		che legge il testo presente in aria-label quando il campo riceve il focus. 
		\item buon contrasto dei colori, verificato per sopperire ai noti problemi dell'ipovedenza.
		\item le parti più rilevanti del sito sono inserite nella comfort zone: per esempio, il menu ad hamburger presente nella versione mobile si 
		posiziona in una zona facilmente raggiungibile dal pollice della mano destra, al pari del contenuto della pagina.\\ 
		(\emph{Nota bene: il form presente nel footer potrebbe essere considerato l'anti-esempio di form inaccessibile, in quanto normalmente i form 
		devono essere posizionati nell'area visibile di un sito. La ragione di questa scelta è che nessun utente, al di fuori dell'amministratore,
		può accedere all'area amministratore. Quindi è sensato inserire il form in una posizione non facilmente visibile all'utente generico, in quanto non gli interessa.}) 
	\end{itemize}
	\item \textbf{Usabilità:} tutti gli utenti possono ottenere quello che vogliono dal sito in un tempo ragionevole e in modo piacevole. E' la conseguenza dell'accessibilità: 
	non ci può essere usabilità senza accessibilità. 
	\item \textbf{Punti di rottura del layout:} applicati per rendere flessibile il sito ai cambiamenti della risoluzione dello schermo, da quelli desktop a quelli mobile.\\
	I punti di rottura sono definiti in css come: \emph{@media only screen and (min-width: ...px) }.
	\item \textbf{Comprensibilità delle informazioni:} le informazioni presentate nel sito sono di semplice comprensione e mirate ad essere esaustive.\\
\end{itemize}